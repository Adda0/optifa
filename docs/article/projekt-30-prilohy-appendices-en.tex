% This file should be replaced with your file with an appendices (headings below are examples only)

% Placing of table of contents of the memory media here should be consulted with a supervisor
\chapter{Contents of the Included Storage Media}

\chapter{Reference Implementation Manual}

In order for reference implementation to work, one requires the following programs: \href{https://www.gnu.org/software/bash/}{Bash}\footnote{\url{https://www.gnu.org/software/bash/}} or any Bash-compatible shell, \href{https://www.python.org/}{Python 3.10}\footnote{\url{https://www.python.org/}} or higher, Python library \href{https://codeberg.org/Adda/symboliclib/}{Symboliclib with our modifications and additions}\footnote{\url{https://codeberg.org/Adda/symboliclib/}; Remember to add Symboliclib to Python path.} and Z3 solver API for Python: Z3Py from \href{https://github.com/Z3Prover/z3}{Z3 solver repository}\footnote{\url{https://github.com/Z3Prover/z3}; Remember to add Z3Py API to Python path.}. Further, to run comparison tests of our optimizations, a command-line benchmarking tool \href{https://github.com/sharkdp/hyperfine}{hyperfine}\footnote{\url{https://github.com/sharkdp/hyperfine}}. The accepted finite automata file format is \href{https://gitlab.inria.fr/regular-pv/timbuk/timbuk/-/wikis/Specification-File-Format}{Timbuk}\footnote{\url{https://gitlab.inria.fr/regular-pv/timbuk/timbuk/-/wikis/Specification-File-Format}}.

Each program can be run with \texttt{-\--help} to show quick help message explaining how to run said program.

You can run tests for all our state language abstractions with \texttt{run\_tests.sh} as follows:
{
    \centering \footnotesize \texttt{./run\_tests.sh -a <finite\_automaton\_A\_file> -b <finite\_automaton\_B\_file> -o <output\_file>}
}

Separate optimizations can be run with their respective scripts:
\begin{itemize}
    \item length abstraction with \texttt{resolve\_satisfiability\_length\_abstraction.py}, and
    \item Parikh image abstraction with \texttt{resolve\_satisfiability\_parikh\_image.py}.
\end{itemize}
Combined optimization algorithm using both length and Parikh image abstractions can be run with \texttt{resolve\_satisfiability\_combined.py}.

Each program offers various flags and required or optional argumements to run adjust the run according to our requirements: Whether to construct full product or just test emptiness of the intersection, which abstraction-specific optimizations to enable, where to store results and the generated product, etc.

Automata with replaced transition by minterms cen be generated with \texttt{get\_minterms.py}.





%\chapter{Configuration file}

%\chapter{Scheme of RelaxNG configuration file}

%\chapter{Poster}

\chapter{Complete Optimization Algorithm}

\begin{algorithm}
\caption{Product construction algorithm using both length abstraction and Parikh image computation and all their optimizations.}
\SetKwInOut{Input}{Input}
\SetKwInOut{Output}{Output}
\SetKwFunction{FStateAbstractionPI}{$ \alpha^{PI} $}
\SetKwFunction{FSMTSolverPush}{smtSolverPush}
\SetKwFunction{FSMTSolverPop}{smtSolverPop}

\SetKwFunction{FIsSkippable}{skippable}
\SetKwFunction{FStateAbstractionLA}{$ \alpha^{LA} $}
\SetKwFunction{FStateAbstractionPI}{$ \alpha^{PI} $}
\SetKwFunction{FAddStateSpecificClauses}{addStateSpecificClauses}
\SetKwFunction{FAddPersistentClauses}{addPersistentClauses}
\SetKwProg{Fn}{Function}{:}{}


\DontPrintSemicolon
\Input{ NFA $A_1 = (Q_1, \Sigma, \delta_1, I_1, F_1)$, NFA $A_2 = (Q_2, \Sigma, \delta_2, I_2, F_2)$}
\Output{ NFA $P = (A_1 \cap A_2) = (Q, \Sigma, \delta, I, F)$ with $L(A_1 \cap A_2) = L(A_1) \cap L(A_2)$}
\BlankLine
$Q, \delta, F \gets \emptyset$ \;
$I \gets I_1 \times I_2$ \;
$W \gets I$ \;
$res \gets False$ \;
$solved \gets \emptyset$ \;
$ \FAddPersistentClauses{} $\;
\While{$W \neq \emptyset$}{
    \textbf{picklast} $[q_1, q_2]$ \textbf{from} $W$ \;
    \textbf{add} $[q_1, q_2]$ \textbf{to} $solved$ \;

    \eIf{$\FIsSkippable{$[q_1, q_2]$}$} {

        $res \gets True$ \;
    } {
        % Check length satisfiability.
        \eIf{$\FStateAbstractionLA{$q_{1}$} \land \FStateAbstractionLA{$q_{2}$} \textbf{ is } \emph{unsat}$}{
            $res \gets False$ \;
        }{ % Else branch.

            \FSMTSolverPush{} \;
            $ \FAddStateSpecificClauses{$[q_1, q_2]$} $\;
            $res \gets \FStateAbstractionPI{$q_{1}$} \land \FStateAbstractionPI{$q_{2}$} \textbf{ is } \emph{sat} $ \;
            \FSMTSolverPop{} \;
            \If{$res = Unknown$}{
                $res \gets True$ \;
            }
        }
    }


    \If{$res = True$}{
        \textbf{add} $[q_1, q_2]$ \textbf{to} $Q$ \;
        \If{$q_1 \in F_1$ \textbf{and} $q_2 \in F_2$} {
        \textbf{add} $[q_1, q_2]$ \textbf{to} $F$ \;
        }
        \ForAll{$a \in \Sigma$}{
            \ForAll{$q'_1 \in \delta_1(q_1, a), q'_2 \in \delta_2(q_2, a)$}{
                \If{$[q'_1, q'_2] \notin solved$ \textbf{and} $[q'_1, q'_2] \notin W$}{
                    \textbf{add} $[q'_1, q'_2]$ \textbf{to} $W$ \;
                }
                \textbf{add} $[q'_1, q'_2] \textbf{ to } \delta([q_1, q_2], a)$ \;
            }
        }
    }
}
\end{algorithm}

